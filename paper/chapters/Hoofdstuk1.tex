\chapter{Literature study}
\label{chap:rel_work}

We will first examine the effectiveness of existing models on a collection of paintings from 2 different movements.
For this we will need to have a pose estimator, a style transformer and a collection of test data. 

A first method: We will first convert the test data with the style transformer to a painting and then we will apply pose estimation.
The test data will have coordinates of the joints, which we will compare with the results of the pose estimation.
However, the joints are of the original image. How do we convert those coordinates to map to the styled image? 
Problem: This method does not use any real paintings and will be susceptible to the accuracy of the style transformer.  

A second method: We can apply pose estimation to real paintings and then convert them to a realistic image with style transfer.
We can then use pose estimation to the realistic images and compare them with the style transformed results.
This will also require a way to map the results of the real painting to that of the style transformed. 
Problem: While we’re using real paintings now, the results will still depend on the accuracy of style transformer. 

A third method: We can annotate the paintings ourselves and use pose estimation to assess the pose estimation algorithms. 
Problem: We must annotate the paintings ourselves. 

\section{Pose estimation}
\label{sec:pose_estimation}

Pose estimation can be achieved in a number of different ways.
There's been 3 methods which we've been able to identify: top-down \cite{alphapose}, bottom-up \cite{blazepose} and instance-aware \cite{fcpose}.
The algorithms are further categorized in single-person \cite{blazepose} and multi-person \cite{alphapose}\cite{fcpose}

\subsection{Functie is\_isbn}