\chapter{Evaluation in the Wild}
This chapter will run the algorithms on the Art Collection from \gls{RMFAB}
as well as some that didn't qualify, but of which the results on a small dataset is still interesting.
From the Art Collection a set of images is chosen that have the highest rate of failure.
These include images with overlapping persons, occlusion, deformation, ...

\section{RMFAB Dataset}
What choices were made to establish the RMFAB dataset

\section{Tests}

Explanation of what tests were run
UGATIT was adapted to randomize the B image in the dataset.
We use the unaligned dataset from CycleGAN to do this

\section{Results}
What are the results from the tests

\section{Discussion}
Is it even possible to encode the information in an image correctly.
When you look at several painting from monet where he draws the same "content" at different times but in the same season there's still a significant difference between them.
It could be that the mood of the artist changed that caused him to choose a different color, or that some lighting or other influences outside the frame change its "style".
Like with Claude Monet, who has many different paintings of the same subject.

