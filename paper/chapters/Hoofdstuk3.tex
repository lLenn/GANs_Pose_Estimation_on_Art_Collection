\chapter{Improving Pose Estimation with Style Transfer}
\label{chap:improvements}
Having established a baseline, it is now possible to search for improvements.
In this chapter, 2 techniques will be explored to see if they can improve \gls{HPE}.
Using the same algorithms as seen in the previous chapter, they will now be used to
(1) transform an input artistic image to a photographic image to estimate poses on or
(2) be trained with a dataset that is augmented with images that are transformed to different styles.

\section{Pose Estimation after Style Transform}
This section will discuss (1)

\section{Augmenting COCO Dataset for Pose Estimation Training}
This section will discuss (2)
Top-down pose estimators also require that the human detector is trained with styled images.

\section{Discussion}
It would be useful to train a network to learn proper 
Suppose you take a \gls{CNN}: it will do convolutions, max pooling until you get as output a vector which you can use for cross-entropy, softmax loss.
This has the entire image as perceptive field, with every layer the perceptive field grows bigger (check if this is true) until the last layer has the entire image in its field.
Suppose that you want to know the coordinates of the object found, all you would need to know is what point in the neural network the perceptive field can see the object.
From that point in the network it would be convenient to have the coordinates marked somewhere so that the object can be found at different scales.
Meaning that for every layer it branches to a subnetwork or as another entrance for backpropogation (as with RNN).
Is this how HRNet works? (research)
Why is dataset thrice the size as the original dataset?
