\chapter{Introduction}
\label{chap:intro}

\section{Problem definition}

To make art collections more accessible, museums put a huge effort in digitalizing their catalogue.
However, they don't contain much metadata about the content and it is time-consuming to enhance them manually.
To make ths process easier, they want to utilize computer vision.
Art collections (paintings, statues, drawings, etc.) turn out to be less interpretable by the algorithms that were developed for photography over the last few decades.
These scan the images in search of recognizable objects and add their labels to the metadata.
Even the latest state-of-the-art technology (Yolov5), struggles to recognize objects when pointed at a painting in a museum.  
A solution may be to start over and have paintings annotated by humans.  
  
This has been done in 2 recent projects:  Saint-George-On-A-Bike \cite{GeorgeOnABike} and INSIGHT \cite{Insight}.
However, paintings are very complex and manual annotation doesn't scale and is very expensive.
For example, 10,000 paintings were annotated by \gls{RMFAB} with no clear return on investment.
They spent a year on this and this is not something they want to repeat.
How can we automate this process and ensure that state-of-the-art computer vision models give good results on paintings and artworks?  
  
Specifically for this thesis, pose estimation will be investigated.

