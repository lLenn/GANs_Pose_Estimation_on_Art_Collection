\chapter*{Abstract}
\chaptermark{Abstract}
\addcontentsline{toc}{chapter}{Abstract}  

Through digitalization, museums are given the ability to more efficiently analyze their art collections.
Important connections between artworks can be uncovered this way, which can be useful for classification or retrieval.
Museums put a great amount of effort in this process, but it can be very labor intensive doing this manually.
To eliminate this issue, they've sought to automate these tasks using computer vision methods.
In computer vision, there's a rich volume of research in image classification, semantic segmentation, object detection and 2D/3D human pose estimation (HPE).
It turns out however, that these algorithms aren't suitable for tasks on art collections as they were trained on photographs.
This thesis will deal with the HPE problem and what methods can be used to improve performance on art collections.
Two shortcomings can be identified: incomplete keypoint prediction and wrong pose association.
To solve this problem, this thesis proposes a method which fine-tunes state-of-the-art (SOTA) HPE models with a combination of stylized COCO datasets.
Three datasets were created from the WikiArt dataset representing baroque, renaissance and impressionism.
From those genres a selection of figurative paintings is made using content-based image retrieval.
Then for each style transfer model, first, a mixture of genres is used and, second, one with only impressionism to create a stylized COCO dataset.
This is done for CycleGAN and AdaIN.
Then, the SWAHR and ViTPose pose estimation models are fine-tuned on the COCO dataset in combination with the stylized COCO dataset, and with only the stylized COCO dataset.
This makes a total of 16 models that are evaluated and in which a consistent improvement in pose estimation prediction was found.

Index terms $-$ Cultural Heritage, Computer Vision, Deep Learning, 2D Human Pose Estimation, Style Transfer, CycleGAN, AdaIN, SWAHR, ViTPose