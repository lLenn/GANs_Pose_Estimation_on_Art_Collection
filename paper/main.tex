%
% Master thesis template for Ghent University (2021)
%
%
%  !!!!!!!!!!!!!!!!!!!!!!!!!!!!!!!!!!!!!!!!!!!!!!!!!!!!!!!!!!!!
%  !!  MAKE SURE TO SET lualatex OR xelatex AS LATEX ENGINE  !!
%  !!!!!!!!!!!!!!!!!!!!!!!!!!!!!!!!!!!!!!!!!!!!!!!!!!!!!!!!!!!!
%  !! For overleaf:                                          !!
%  !!     1. click gear icon in top right                    !!
%  !!     2. select `lualatex` in "latex engine"             !!
%  !!     3. click "save project settings"                   !!
%  !!                                                        !!
%  !!!!!!!!!!!!!!!!!!!!!!!!!!!!!!!!!!!!!!!!!!!!!!!!!!!!!!!!!!!!
%
%
%  History
%    2014         Doctoral Thesis of Bruno Volckaert
%    2017         Adapted to master thesis by Jerico Moeyersons
%    2018         Cleanup by Merlijn Sebrechts
%    2021         Update by Marleen Denert and Merlijn Sebrechts with feedback from Leen Pollefliet
%
%  Latest version
%    https://github.com/galgalesh/masterproef-template
%
\documentclass[11pt,a4paper,openany]{book}
\usepackage[a4paper,includeheadfoot,margin=2.50cm]{geometry}

\renewcommand{\baselinestretch}{1.2}  % stretch horizontal space between everything

\usepackage[hyphens]{url} % Break line on hyphens in long urls
\usepackage{graphicx}
\graphicspath{{images/}}
\usepackage{pdfpages}
\usepackage{enumitem}
\usepackage{float}
\usepackage{caption}
\usepackage{subcaption}
\usepackage[toc,page]{appendix}
\usepackage{fontspec}
\usepackage[T1]{fontenc}

% Don't indent table of contents, list of figures, and list of tables
\usepackage{tocloft}
\setlength{\cftsecindent}{0pt}    % Remove indent for \section
\setlength{\cftsubsecindent}{0pt} % Remove indent for \subsection
\setlength{\cftfigindent}{0pt}    % remove indentation from figures in lof
\setlength{\cfttabindent}{0pt}    % remove indentation from tables in lot

% To generate fake lorem ipsum text
\usepackage{lipsum}



%
% UGent style guide
%
\setmainfont[
	Path=fonts/,
	BoldFont      =UGentPannoText-SemiBold.ttf,
	ItalicFont    =UGentPannoText-Normal.ttf,
	ItalicFeatures={FakeSlant=0.3},
	BoldItalicFont=UGentPannoText-SemiBold.ttf,
    BoldItalicFeatures={FakeSlant=0.3},
]{UGentPannoText-Normal.ttf}
\urlstyle{same} % Also use the default font for URLs


% If you want left justified text, uncomment the line below.
%\usepackage[document]{ragged2e} % Left justify all text

% Style Chapter titles so they have the chapter number in grey.
\usepackage{color}
\definecolor{chaptergrey}{rgb}{0.5,0.5,0.5}
\usepackage[explicit, pagestyles]{titlesec}
\titleformat{\chapter}[display]{\bfseries}{\color{chaptergrey}\fontfamily{pbk}\fontsize{80pt}{100pt}\selectfont\thechapter}{0pt}{\Huge #1}
\titlespacing*{\chapter}{0pt}{-80pt}{30pt}


% Header showing chapter number and title and footer showing page number
\newpagestyle{fancy}{%
  \sethead{} % left
          {} % center
          {\Large\thechapter~~\chaptertitle} %right
  \setfoot{} % left
          {\thepage} % center
          {} %right
  \setheadrule{0pt}
}
\pagestyle{fancy}

% Header showing chapter title and footer showing page number
\newpagestyle{numberless}{%
  \sethead{} % left
          {} % center
          {\Large\chaptertitle} %right
  \setfoot{} % left
          {\thepage} % center
          {} %right
  \setheadrule{0pt}
}

% We use the package `minted` for modern code highlighting.
\usepackage[newfloat,chapter]{minted}
\SetupFloatingEnvironment{listing}{name=Code Fragment, listname=List of Code Fragments} % lang:english


\PassOptionsToPackage{hyphens}{url}
\usepackage{hyperref}
\usepackage{url}

\usepackage[numbers]{natbib}       % For bibliography; use numeric citations
\bibliographystyle{IEEEtran}
\usepackage[nottoc]{tocbibind}     % Put Bibliography in ToC

%
% Defines \checkmark to draw a checkmark
%
\usepackage{tikz}
\def\checkmark{\tikz\fill[scale=0.4](0,.35) -- (.25,0) -- (1,.7) -- (.25,.15) -- cycle;}

%
% For tables
%
\usepackage{booktabs}
\usepackage{array}
\usepackage{ragged2e}  % for '\RaggedRight' macro (allows hyphenation)
\newcolumntype{L}[1]{>{\raggedright\let\newline\\\arraybackslash\hspace{0pt}}m{#1}}
\newcolumntype{C}[1]{>{\centering\let\newline\\\arraybackslash\hspace{0pt}}m{#1}}
\newcolumntype{R}[1]{>{\raggedleft\let\newline\\\arraybackslash\hspace{0pt}}m{#1}}

%
% Support for splitting English words correctly
%
\usepackage{polyglossia}
\setdefaultlanguage[variant=us]{english}

\renewcommand\appendixtocname{Bijlagen}
\renewcommand\appendixpagename{Bijlagen}

\usepackage[toc,acronym]{glossaries}  % for list of acronyms
\makeglossaries                       % start internal list of acronyms

%
% Set the title and your name
%
%%%%%%%%%%%%%%%%%%%%%%%%%%%%%%%%%%%%%%%%%%%%%%%%%%%%%%%%%%%%%%%%%%%%%%
%
% Add the specific info for your thesis
%
%%%%%%%%%%%%%%%%%%%%%%%%%%%%%%%%%%%%%%%%%%%%%%%%%%%%%%%%%%%%%%%%%%%%%%

\title{Using GANs to optimize Pose Estimation on Art Collections}
\author{Tristan Verheecke}

%%%%%%%%%%%%%%%%%%%%%%%%%%%%%%%%%%%%%%%%%%%%%%%%%%%%%
% Add all the acronyms you use in your thesis here. %
% These will be added to the List of Acronyms       %
%%%%%%%%%%%%%%%%%%%%%%%%%%%%%%%%%%%%%%%%%%%%%%%%%%%%%

\newacronym{RMFAB}{RMFAB}{Royal Museums of Fine Arts of Belgium}
\newacronym{HPE}{HPE}{Human Pose Estimation}
\newacronym{CNN}{CNN}{Convolutional Neural Network}
\newacronym{MPII}{MPII}{Max Planck Institute for Informatics}
\newacronym{ASMs}{ASMs}{Active Shape Models}
\newacronym{SMPL}{SMPL}{Skinned Multi-Person Linear}

%
%  END OF HEADER
%  The actual latex document content starts here.
%
\begin{document}
\frontmatter
\pagestyle{empty}
\includepdf{voorblad.pdf}

\chapter*{Preface}

I've been interested in Art my entire life. In fact, I've a degree in the Fine Arts from LUCA School of Arts.
There, I was known for my technological ability and one of my professors at the time asked me why I didn't do anything with that in my artworks.
That remark has since stuck with me and was part of my motivation to apply for readmission for my Master of Science.
With all the advancements in AI, I started thinking more and more about doing work with that.
Like Matisse and Turner, I'm not satisfied with the tools available, but want to create my own.

It was therefor to my delight that I was able to work on this thesis which has provided me the opportunity to acquire more insight in the subject.
I would like to thank my supervisors Dieter De Witte and Steven Verstockt for this wonderful opportunity,
and my counsellor Kenzo Milleville for his great guidance. As well as all the other people at IDLab for their feedback.
I also want to thank Karine Lacaracina, Lies Van De Cappelle and the other people at RMFAB for providing help with the artistic sensibilities of the thesis.
\\
\\
Enjoy the read,
\\
\\
Tristan Verheecke\\
Ghent, June 2023

% Om het extended abstract te schrijven kan je de IEEE conference proceedings template gebruiken. Die staat ook op Overleaf: https://www.overleaf.com/latex/templates/ieee-conference-template/grfzhhncsfqn. Voeg dit toe als .pdf
\includepdf[pages={-}]{abstract.pdf}  % Extended Abstract

\tableofcontents                      % Table of Contents
\newpage
\listoffigures                        % List of figures
\newpage
\listoftables                         % List of tables
\newpage
\include{chapters/Acronyms}           % List of acronyms
\listoflistings                       % List of listings (code fragments)
\newpage

%
% Include the main chapters of the thesis below
%
\mainmatter
\pagestyle{empty}
\chapter{Introduction}
\label{chap:intro}

\section{Problem definition}

Part of the modern age is the digitalization of information.
Digitization makes information more accessible to a broader audience and allows it to be processed more efficiently.
Museums have put huge efforts in digitalizing their catalogue.
While viewing the original artworks firsthand still provides a unique experience, the rarest and most beloved works are sealed behind glass.
Through digitalization, a museum can offer people the experience of viewing them from up close.
It also allows them to make more of their collection available as now they can only show a small percentage of their work at once.
Some people might never have the chance to see these works, but through digitalization this becomes possible \cite{Digitization1}.

Digitization can also help in Iconography; this is the branch of art history that concerns itself with the themes and motifs of artworks.
Through the analysis of artworks, different connections between different artworks can be established, which can be useful for classification or retrieval.
However, art collections don't contain much metadata and it is time-consuming to enhance them manually.
Museums want to utilize computer vision to automate this process, but the algorithms that were developed over the last few decades, are mainly for photography and it turns out that art collections (paintings, statues, drawing, etc) are less interpretable by these algorithms.
These algorithms scan the images in search of recognizable objects and add their labels to the metadata.
Even the latest state-of-the-art technology, struggles to recognize objects when pointed at a painting in a museum.

A solution may be to start over and have paintings be annotated by humans.
This has been done in two recent projects: Saint-George-On-A-Bike \cite{GeorgeOnABike} and INSIGHT \cite{Insight}.
However, paintings are very complex while manual annotation doesn't scale and is very expensive.
For example, 10,000 paintings were annotated by the \gls{RMFAB} with no clear return on investment \cite{RMFAB2024}.
They spent a year on this and this is not something they want to repeat.
How can we automate this process and ensure that state-of-the-art computer vision models give good results on paintings and artworks?  

Computer vision can perform a wide range of tasks, including image classification, semantic segmentation, object detection, and 2D/3D \gls{HPE}.
For this thesis, the focus will be on 2D HPE.
A database can be created with the different poses found in the artworks which can be used to discover similar themes and categorize them.
There is an extensive amount of research based around HPE that can be useful for this.
\\
\\

\section{Proposed solution}

To make the vast quantity of research around HPE available to art collections, there are two proposed solutions that will be explored.
The efficacy of these methods wil be analyzed in this thesis.

\begin{enumerate}
    \item The input artwork is first converted to photographic realism on which pose estimation is then executed.
With this method the pre-trained models from the state-of-the-art architectures can be reused without need to do any new adjustments.
This method does require the style transfer network to have a high fidelity to realism.
    \item If the pre-trained models can't be used, it's still an option to retrain one with an augmented dataset.
With style transfer the images of existing datasets can be stylized and added to the datasets.
This will increase the size and variance of the dataset, making it better to train on.
This can increase performance on art collections as stylized images are also being trained on but can also potentially increase the performance on photographs.
\end{enumerate}

\section{Thesis Outline}
Chapter \ref{chap:rel_work} discusses the related literature.
The research in style transfer is looked at in depth, as well as the extensive body of work in pose estimation.
Content-Based Image Retrieval is shortly described and the chapter closes with a look at deep learning on art collections.
\\
\\
Chapter \ref{chap:style_transfer} discusses the use of different style transfer models.
Novel datasets are created to train several new models.
The models are evaluated with different evaluation metrics and a selection of the best is made.
\\
\\
Chapter \ref{chap:baseline} discusses the creation of a baseline for the experiments.
Two different methods are used to achieve this.
\\
\\
Chapter \ref{chap:improvements} discusses two improvement proposals.
The first transforms the input for existing pose estimation models to photorealistic images.
The second augments the COCO-dataset with style transfer to train new pose estimation models.
Afterwards, the results are discussed.
\\
\\
Chapter \ref{chap:conclusion} wraps up the thesis with a summary and suggestions for future research.

\chapter{Literature study}
\label{chap:rel_work}

We will first examine the effectiveness of existing models on a collection of paintings from 2 different movements.
For this we will need to have a pose estimator, a style transformer and a collection of test data. 

A first method: We will first convert the test data with the style transformer to a painting and then we will apply pose estimation.
The test data will have coordinates of the joints, which we will compare with the results of the pose estimation.
However, the joints are of the original image. How do we convert those coordinates to map to the styled image? 
Problem: This method does not use any real paintings and will be susceptible to the accuracy of the style transformer.  

A second method: We can apply pose estimation to real paintings and then convert them to a realistic image with style transfer.
We can then use pose estimation to the realistic images and compare them with the style transformed results.
This will also require a way to map the results of the real painting to that of the style transformed. 
Problem: While we’re using real paintings now, the results will still depend on the accuracy of style transformer. 

A third method: We can annotate the paintings ourselves and use pose estimation to assess the pose estimation algorithms. 
Problem: We must annotate the paintings ourselves. 

\section{Pose estimation}
\label{sec:pose_estimation}

Pose estimation can be achieved in a number of different ways.
There's been 3 methods which we've been able to identify: top-down \cite{alphapose}, bottom-up \cite{blazepose} and instance-aware \cite{fcpose}.
The algorithms are further categorized in single-person \cite{blazepose} and multi-person \cite{alphapose}\cite{fcpose}

\subsection{Functie is\_isbn}
\chapter{Establishing a Baseline}
\label{chap:baseline}
This chapter will establish the baseline that will be used to compare our results with.
For this, 2 to 3 algorithms from both pose estimation and style transfer will be explored.
The motivation for the choices of the algorithms will be explained in full detail.
First, style transfer will be applied to the COCO dataset to then estimate any poses from it.
The results wil give an indication of how well pose estimation will work on art collections.
More recently, a new dataset has emerged which will be of great help, the Human-Art dataset \cite{Ju2023}, with which we can directly check the pose estimation without an intermediary step.

\section{Choice of Pose Estimation}
\label{sec:baseline_pose_estimation}
(dirty version)
There are a few choice that are evident, does it have code available and is it compatible with the chosen dataset.
Another is time of inference, how fast can it estimate the pose? This paper doesn't need real-time inference, but a algorithm can both be fast and accurate \cite{William2021}
Want to explore a diverse set of estimators. (bottom-up, top-down, ...)
(SWAHR explain why...) \cite{SWARH} Faster network according to surveys. Uses the popular network HRNet. (Bottom-Up)
(KAPAO explain why...) \cite{William2021} Claims to be both fast and accurate. (Single-stage; explain single stage in literature study. It means that it does away with the top-down/bottom-up paradigm which are two-stage models.)
(VitPose explain why...) \cite{vitpose} Uses transformers

\section{Choice of Style Transfer}
\label{sec:baseline_style_transfer}
(dirty version)
A similar criteria as or pose estimation applies: does it have code, time of transformation
Uniquely: does it have pretrained models for the styles we want?
Does it apply transformation? (U-GAT-IT) (We don't want transformation, but interesting for future research)
(CycleGAN ...) \cite{Zhu2017} Has the most pre-trained art models available
(UNIT or StarGANv2 ...) \cite{Liu2017} Latent-space but no pretrained artistic model, but can we initialize weights with other models to speed up training?
1 - photo, 2 - baroque, 3 - impressionism, 4 - renaissance
(AdaIn) 
Another possible way to speed up training is to focus the dataset on human poses.
Which is why image retrieval has been discussed previously.
This way we can extract have more specialized datasets from the existing datasets.
Even with the genre categorization it's still too broad.
This has become apparent when training U-GAT-IT (This was before I realized that this model also does content transformation)
(StyleGAN) Because I feel that the images to be trained on are not genre or artist dependent, because there is still a great variety among them.
I think that it should be possible to train a style transfer with only a few samples.
It was said to me that StyleGAN did this.
I also want to research diffusion.

StarGAN experiences mode collapse after 100,000 iterations. (iterations because their implementation doesn't use epochs)
\section{Choice of Dataset}
(dirty version)
CIBR works well when there's something very recognisable, like a tennis court.
It has come to my attention that making a distinction between real life and art for style transfer is a mistake.
Viewing only art as having styles is a mistake.
Real life can have just as many different style.
Whether it is style of clothing, lighting or camera filter, while at the same time being possibly content.
The natural day and night cycle should be considered content, but artificial light should be considered style.
\subsection{Finding a good query image}
Several import aspects should be considered when searching for a good query image.
Does it contain the right kind of content and no other content to distract, like a car in the background or even a small flower in the foreground.
Should it be an image from the style we're trying to transform to?
The only images that yielded similar persons with different poses were when we queried an image with sports, like tennis.
Maybe an "instance image retrieval" algorithm is not the right algorithm? We could maybe get better results with a "category retrieval algorithm"?

I removed masking from the COCO dataset which was a mistake. 
When training a top-down estimator, then how does it make a destinction between people in a bounding box that are in the background?
How many people will it find that way?

\section{Pose Estimation after Applying Style Transfer to the COCO Dataset}
\label{sec:baseline_coco_style_transfer}

\subsection{Architecture}
\label{sec:baseline_coco_style_transfer_architecture}
Needed to implement a script that transforms a training model to a test model because CycleGAN leaves out certain layers for test, like dropout.

\subsection{Results}
How good is this as a measurement of the pose estimator?
It's entirely reliable on the style transfer.
For the CycleGAN model, there are some photo's that change little after applying style transfer.
For SWAHR, forgot to add masks in dataset (removed them while coping because didn't think it relevant, but there are images with multiple people where some people are not in the keypoint instances, so they need to be removed)
Only found out about this when a masked image was shown in the experiments
For ViTPose or other top-down pose estimators, the cropping of the image removes a lot of information that could potentially be relevant, like sitting on the back of an elephant or perspective.
Can a network even look for perspective and environmental clues in the first place?
Perhaps 2d is limited in that sense?
We consider the tests on 2 datasets: 
1. A set specifically made to measure style transfer.\cite{Chen2023} \cite{ioannou2024} Which contains a range of images over a different range of "measurements" 
Our own dataset of people ...
2. Custom dataset is create by running CIRQ on coco dataset with following query images: First query gives good results, but are dominated by outdoor images.
The following queries were done to find more images in a different context
While in the original paper there is only one style image used to train the network, here we compare the used content image and style image used for each transfer.
For CycleGAN and StarGAN, we cycle through the real images to determine the style loss.
\begin{table*}
    \setlength\tabcolsep{4pt}
    \vspace{0.2em}
    \caption{Performance comparison of Style Transfer measured by various metrics; Perceptual Distance (PD), Inception score (IS), Fréchet Inception Distance (FID) and Learned Perceptual Image Patch Similarity (LPIPS). }
    \centering
    \footnotesize
    \label{tab:performance_style_transfer}
    \begin{tabular}{ l|cccc|cccc|cccc }
        \hline
        \multirow{2}{*}{\bf{Method}}&\multicolumn{4}{c|}{\bf{Baroque}}&\multicolumn{4}{c|}{\bf{Renaissance}}&\multicolumn{4}{c}{\bf{Impressionism}}\cr
        &\bf{PD}&\bf{IS}&\bf{FID}&\bf{LPIPS}&\bf{PD}&\bf{IS}&\bf{FID}&\bf{LPIPS}&\bf{PD}&\bf{IS}&\bf{FID}&\bf{LPIPS}\cr
        \hline
        \multicolumn{13}{c}{\bf{AST-IQAD Database}}\cr
        \hline
        CycleGAN & 0 & 0 & 0 & 0 & 0 & 0 & 0 & 0 & 0 & 0 & 0 & 0 \cr
        AdaIN & 0 & 0 & 0 & 0 & 0 & 0 & 0 & 0 & 0 & 0 & 0 & 0 \cr
        StarGAN & 0 & 0 & 0 & 0 & 0 & 0 & 0 & 0 & 0 & 0 & 0 & 0 \cr
        \hline 
        \multicolumn{13}{c}{\bf{Custom Database}}\cr
        \hline
        CycleGAN & 0 & 0 & 0 & 0 & 0 & 0 & 0 & 0 & 0 & 0 & 0 & 0 \cr
        AdaIN & 0 & 0 & 0 & 0 & 0 & 0 & 0 & 0 & 0 & 0 & 0 & 0 \cr
        StarGAN & 0 & 0 & 0 & 0 & 0 & 0 & 0 & 0 & 0 & 0 & 0 & 0 \cr
        \hline 
    \end{tabular}
\end{table*}

\label{sec:baseline_coco_style_transfer_results}

\section{Pose Estimation on the Human-Art Dataset}
\label{sec:baseline_human_art}

\subsection{Architecture}
\label{sec:baseline_human_art_architecture}

\subsection{Results}
\label{sec:baseline_human_art_results}

SWAHR and HumanArt Dataset (validation set w32\_512)\\
Average Precision  (AP) @[ IoU=0.50:0.95 | area=   all | maxDets= 20 ] = 0.469\\
 Average Precision  (AP) @[ IoU=0.50      | area=   all | maxDets= 20 ] = 0.688\\
 Average Precision  (AP) @[ IoU=0.75      | area=   all | maxDets= 20 ] = 0.499\\
 Average Precision  (AP) @[ IoU=0.50:0.95 | area=medium | maxDets= 20 ] = 0.066\\
 Average Precision  (AP) @[ IoU=0.50:0.95 | area= large | maxDets= 20 ] = 0.512\\
 Average Recall     (AR) @[ IoU=0.50:0.95 | area=   all | maxDets= 20 ] = 0.529\\
 Average Recall     (AR) @[ IoU=0.50      | area=   all | maxDets= 20 ] = 0.726\\
 Average Recall     (AR) @[ IoU=0.75      | area=   all | maxDets= 20 ] = 0.562\\
 Average Recall     (AR) @[ IoU=0.50:0.95 | area=medium | maxDets= 20 ] = 0.111\\
 Average Recall     (AR) @[ IoU=0.50:0.95 | area= large | maxDets= 20 ] = 0.573\\
| Arch | AP | Ap .5 | AP .75 | AP (M) | AP (L) | AR | AR .5 | AR .75 | AR (M) | AR (L) |\\
|---|---|---|---|---|---|---|---|---|---|---|\\
| SWAHR | 0.469 | 0.688 | 0.499 | 0.066 | 0.512 | 0.529 | 0.726 | 0.562 | 0.111 | 0.573 |\\

SWAHR and HumanArt Dataset (validation set w48\_640)\\
Average Precision  (AP) @[ IoU=0.50:0.95 | area=   all | maxDets= 20 ] = 0.494\\
Average Precision  (AP) @[ IoU=0.50      | area=   all | maxDets= 20 ] = 0.705\\
Average Precision  (AP) @[ IoU=0.75      | area=   all | maxDets= 20 ] = 0.526\\
Average Precision  (AP) @[ IoU=0.50:0.95 | area=medium | maxDets= 20 ] = 0.083\\
Average Precision  (AP) @[ IoU=0.50:0.95 | area= large | maxDets= 20 ] = 0.538\\
Average Recall     (AR) @[ IoU=0.50:0.95 | area=   all | maxDets= 20 ] = 0.556\\
Average Recall     (AR) @[ IoU=0.50      | area=   all | maxDets= 20 ] = 0.749\\
Average Recall     (AR) @[ IoU=0.75      | area=   all | maxDets= 20 ] = 0.592\\
Average Recall     (AR) @[ IoU=0.50:0.95 | area=medium | maxDets= 20 ] = 0.149\\
Average Recall     (AR) @[ IoU=0.50:0.95 | area= large | maxDets= 20 ] = 0.600\\
| Arch | AP | Ap .5 | AP .75 | AP (M) | AP (L) | AR | AR .5 | AR .75 | AR (M) | AR (L) |\\
|---|---|---|---|---|---|---|---|---|---|---|\\
| SWAHR | 0.494 | 0.705 | 0.526 | 0.083 | 0.538 | 0.556 | 0.749 | 0.592 | 0.149 | 0.600 |\\

\begin{table*}
    \setlength\tabcolsep{4pt}
    \caption{
        Establishing a baseline for Pose Estimation on Artworks; measuring Percentage of Correct Keypoints (PCK) and Average Precision/Recall (AP/AR).
        The first section shows the performance of the plain models on The COCO dataset measured.
        The second shows the performance of the different models on the Human-Art dataset.
    }
    \centering
    \footnotesize
    \label{tab:baseline_pose_estimation_after_style_transfer}
    \begin{tabular}{ l|cc|ccccc|ccccc }
        \hline
        \bf{Method}&\bf{PCK@0.2}&\bf{PCKh@0.5}&\bf{AP}&\bf{AP$^{50}$}&\bf{AP$^{75}$}&\bf{AP$^{M}$}&\bf{AP$^{L}$}&\bf{AR}&\bf{AR$^{50}$}&\bf{AR$^{75}$}&\bf{AR$^{M}$}&\bf{AR$^{L}$}\cr
        \hline
        \multicolumn{13}{c}{\bf{COCO dataset}}\cr
        \multicolumn{13}{c}{\bf{Trained on COCO}}\cr
        \hline
        SWAHR & 0 & 0 & 0 & 0 & 0 & 0 & 0 & 0 & 0 & 0 & 0 & 0 \cr
        ViTPose & 0 & 0 & 0 & 0 & 0 & 0 & 0 & 0 & 0 & 0 & 0 & 0 \cr
        \hline
        \multicolumn{13}{c}{\bf{Human-Art Dataset}}\cr
        \multicolumn{13}{c}{\bf{Trained on COCO}}\cr
        \hline
        SWAHR & 0 & 0 & 0 & 0 & 0 & 0 & 0 & 0 & 0 & 0 & 0 & 0 \cr
        ViTPose & 0 & 0 & 0 & 0 & 0 & 0 & 0 & 0 & 0 & 0 & 0 & 0 \cr
        \hline
        \multicolumn{13}{c}{\bf{Trained on COCO + Mixed Style Transfer}}\cr
        \hline
        SWAHR & 0 & 0 & 0 & 0 & 0 & 0 & 0 & 0 & 0 & 0 & 0 & 0 \cr
        ViTPose & 0 & 0 & 0 & 0 & 0 & 0 & 0 & 0 & 0 & 0 & 0 & 0 \cr
        \hline
        \multicolumn{13}{c}{\bf{Trained on COCO + Impressionism Style Transfer}}\cr
        \hline
        SWAHR & 0 & 0 & 0 & 0 & 0 & 0 & 0 & 0 & 0 & 0 & 0 & 0 \cr
        ViTPose & 0 & 0 & 0 & 0 & 0 & 0 & 0 & 0 & 0 & 0 & 0 & 0 \cr
        \hline
        \multicolumn{13}{c}{\bf{Trained on Mixed Style Transfer}}\cr
        \hline
        SWAHR & 0 & 0 & 0 & 0 & 0 & 0 & 0 & 0 & 0 & 0 & 0 & 0 \cr
        ViTPose & 0 & 0 & 0 & 0 & 0 & 0 & 0 & 0 & 0 & 0 & 0 & 0 \cr
        \hline
        \multicolumn{13}{c}{\bf{Trained on Impressionism Style Transfer}}\cr
        \hline
        SWAHR & 0 & 0 & 0 & 0 & 0 & 0 & 0 & 0 & 0 & 0 & 0 & 0 \cr
        ViTPose & 0 & 0 & 0 & 0 & 0 & 0 & 0 & 0 & 0 & 0 & 0 & 0 \cr
        \hline
    \end{tabular}
\end{table*}


\begin{table*}
    \setlength\tabcolsep{4pt}
    \caption{
        Establishing a baseline for Pose Estimation on Artworks; measuring Percentage of Correct Keypoints (PCK) and Average Precision/Recall (AP/AR).
        The COCO dataset is transformed with various Style Transfer models on which performance is measured. }
    \centering
    \footnotesize
    \label{tab:baseline_pose_estimation_after_style_transfer}
    \begin{tabular}{ l|cc|ccccc|ccccc }
        \hline
        \bf{Method}&\bf{PCK@0.2}&\bf{PCKh@0.5}&\bf{AP}&\bf{AP$^{50}$}&\bf{AP$^{75}$}&\bf{AP$^{M}$}&\bf{AP$^{L}$}&\bf{AR}&\bf{AR$^{50}$}&\bf{AR$^{75}$}&\bf{AR$^{M}$}&\bf{AR$^{L}$}\cr
        \hline
        \multicolumn{13}{c}{\bf{CycleGAN}}\cr
        \multicolumn{13}{c}{\bf{Trained on COCO + Mixed Style Transfer}}\cr
        \hline
        SWAHR & 0 & 0 & 0 & 0 & 0 & 0 & 0 & 0 & 0 & 0 & 0 & 0 \cr
        ViTPose & 0 & 0 & 0 & 0 & 0 & 0 & 0 & 0 & 0 & 0 & 0 & 0 \cr
        \hline
        \multicolumn{13}{c}{\bf{Trained on COCO + Impressionism Style Transfer}}\cr
        \hline
        SWAHR & 0 & 0 & 0 & 0 & 0 & 0 & 0 & 0 & 0 & 0 & 0 & 0 \cr
        ViTPose & 0 & 0 & 0 & 0 & 0 & 0 & 0 & 0 & 0 & 0 & 0 & 0 \cr
        \hline
        \multicolumn{13}{c}{\bf{Trained on Mixed Style Transfer}}\cr
        \hline
        SWAHR & 0 & 0 & 0 & 0 & 0 & 0 & 0 & 0 & 0 & 0 & 0 & 0 \cr
        ViTPose & 0 & 0 & 0 & 0 & 0 & 0 & 0 & 0 & 0 & 0 & 0 & 0 \cr
        \hline
        \multicolumn{13}{c}{\bf{Trained on Impressionism Style Transfer}}\cr
        \hline
        SWAHR & 0 & 0 & 0 & 0 & 0 & 0 & 0 & 0 & 0 & 0 & 0 & 0 \cr
        ViTPose & 0 & 0 & 0 & 0 & 0 & 0 & 0 & 0 & 0 & 0 & 0 & 0 \cr
        \hline
        \multicolumn{13}{c}{\bf{AdaIN}}\cr
        \multicolumn{13}{c}{\bf{Trained on COCO + Mixed Style Transfer}}\cr
        \hline
        SWAHR & 0 & 0 & 0 & 0 & 0 & 0 & 0 & 0 & 0 & 0 & 0 & 0 \cr
        ViTPose & 0 & 0 & 0 & 0 & 0 & 0 & 0 & 0 & 0 & 0 & 0 & 0 \cr
        \hline
        \multicolumn{13}{c}{\bf{Trained on COCO + Impressionism Style Transfer}}\cr
        \hline
        SWAHR & 0 & 0 & 0 & 0 & 0 & 0 & 0 & 0 & 0 & 0 & 0 & 0 \cr
        ViTPose & 0 & 0 & 0 & 0 & 0 & 0 & 0 & 0 & 0 & 0 & 0 & 0 \cr
        \hline
        \multicolumn{13}{c}{\bf{Trained on Mixed Style Transfer}}\cr
        \hline
        SWAHR & 0 & 0 & 0 & 0 & 0 & 0 & 0 & 0 & 0 & 0 & 0 & 0 \cr
        ViTPose & 0 & 0 & 0 & 0 & 0 & 0 & 0 & 0 & 0 & 0 & 0 & 0 \cr
        \hline
        \multicolumn{13}{c}{\bf{Trained on Impressionism Style Transfer}}\cr
        \hline
        SWAHR & 0 & 0 & 0 & 0 & 0 & 0 & 0 & 0 & 0 & 0 & 0 & 0 \cr
        ViTPose & 0 & 0 & 0 & 0 & 0 & 0 & 0 & 0 & 0 & 0 & 0 & 0 \cr
        \hline
    \end{tabular}
\end{table*}

\section{Related Papers}
Enhancing Human Pose Estimation in Ancient Vase Paintings via Perceptually-grounded Style Transfer Learning \cite{Madhu2020}\\
\subsection{Results}
\label{sec:baseline_related_papers_results}
Compare results with related paper

Discuss the code and discussions during implementation.
Also, what could be done differently (own implementation)
Discuss how there are many different ways to "choose" the style (change model (cyclegan), choose number (stargan), use style image)
This could be solved by creating a new interface for the styles with each their own options, etc 
Make everything highly configurable

With all style transfer, I notice that when trying to convert to photo, they always seem to confuse foreground and background.
You can tell from the stargan training that this seems to be solely with photo.
This could be because of my dataset, but it can also be that paintings aren't as high contrast and lines between foreground and background are less vage.
(check paper about making distinction between foreground and background)

The dataset could have more images, but just not more paintings, but also more different faces and angles of the body, etc, positioned all over the image.
Also examine how the perception field works, like how does it actually work?

Style transfer encoders lose data, one should wonder then whether this is actually a good approach.
Should there instead be more information "encoded" then is visible on the image.
The network should basically be able to render the "content" in 3d.

\section{Discussion}
\label{sec:baseline_discussion}
Already it is apparent that pose estimation on art collections is strongly dependent on the efficacy of the style transfer.
Use something else than visdom, because it takes too long to save when there are too many datapoints.

\chapter{Titel derde hoofdstuk}

Vul aan.


\lipsum[20-24]

\section{Sectie titel}

Vul aan.

\lipsum[32-34]
\section{Sectie titel2}

\lipsum[6-8]

Voorbeeld figuur.

\subsection*{Subtitel}
\addcontentsline{toc}{subsection}{Subtitel}  
Vul aan
\subsection{Functie is\_isbn}

Voorbeeld listing.

\begin{listing}[!h]
\inputminted{python}{isbn.py}
\caption{Functie is\_isbn}
\end{listing}
\chapter{Evaluation in the Wild}
This chapter will run the algorithms on the Art Collection from \gls{RMFAB}
as well as some that didn't qualify, but of which the results on a small dataset is still interesting.
From the Art Collection a set of images is chosen that have the highest rate of failure.
These include images with overlapping persons, occlusion, deformation, ...

\section{RMFAB Dataset}
What choices were made to establish the RMFAB dataset

\section{Tests}

Explanation of what tests were run
UGATIT was adapted to randomize the B image in the dataset.
We use the unaligned dataset from CycleGAN to do this

\section{Results}
What are the results from the tests

\section{Discussion}
Is it even possible to encode the information in an image correctly.
When you look at several painting from monet where he draws the same "content" at different times but in the same season there's still a significant difference between them.
It could be that the mood of the artist changed that caused him to choose a different color, or that some lighting or other influences outside the frame change its "style".
Like with Claude Monet, who has many different paintings of the same subject.

Talk a bit about HD pictures and models


Discuss code:

Discuss the code and discussions during implementation.
Also, what could be done differently (own implementation)
Discuss how there are many different ways to "choose" the style (change model (cyclegan), choose number (stargan), use style image)
This could be solved by creating a new interface for the styles with each their own options, etc 
Make everything highly configurable

Transformation interesting for future research.


Suppose you take a \gls{CNN}: it will do convolutions, max pooling until you get as output a vector which you can use for cross-entropy, softmax loss.
This has the entire image as perceptive field, with every layer the perceptive field grows bigger (check if this is true) until the last layer has the entire image in its field.
Suppose that you want to know the coordinates of the object found, all you would need to know is what point in the neural network the perceptive field can see the object.
From that point in the network it would be convenient to have the coordinates marked somewhere so that the object can be found at different scales.
Meaning that for every layer it branches to a subnetwork or as another entrance for backpropogation (as with RNN).
Is this how HRNet works? (research)
Why is dataset thrice the size as the original dataset?




Discuss the flaws of mmpose: log\_processor doesn't give enough info, eval code during runtime, all centered around configuration, but misses ease of programming.
HAs train\_loop, val\_loop variables for extra confusion
Don't have your code add prefixes to output dirs or anything else. It only causes confusion.
It's also difficult to add new stuff, because the hooks don't provide enough information meaning hacks need to be implemented.
When resuming a network, the iterations continue from previous session, but if the new session has a bigger or smaller world, those iterations don't match the new world size.
How does multiple distribution sessions work?



On how to do research: would a method of research like gradient descent where one does a quick research paper of to check improvements and only proceeds in a certain direction when improvements are significant.
Instead of researching every single variable.



Use a different backend and not visdom. It's difficult to alter test results with visdom, like removing unnecessary domains.

float16 for quicker warmup

When comparing ViTPose and SWAHR we should note that there is a HRNet version with vision transformers and maybe even do some tests.

\chapter{Conclusions}
\label{chap:conclusion}

Because of the digitalization of art collections, museums are looking to improve their analytic tools to help them with their functions.
Among them are the relationships between artworks depending on their themes of which poses are a big part.
Unfortunately, the state-of-the-art pose estimation models have only been trained on photograph datasets and have a miserable performance on art collections.
To achieve better results two methods of improvement were explored:
\begin{enumerate}
    \item The input images for the pose estimation networks are first transformed from an artwork to a photograph.
    With this method, the already vast library of pose estimation methods is made available to the curators.
    \item The COCO dataset is transformed with multiple style transfer methods to styled COCO datasets.
    With these styled datasets, new pose estimation models can be trained which work better than the already existing ones.
\end{enumerate}
These methods require a style transfer method that is able to transform between artworks and photographs.
Therefor, several datasets were created by using CBIR on the WikiArt dataset.
The focus of these datasets is mainly around the human figure as this is the domain of pose estimation.
Several style transfer models were trained, but it was found that they did not achieve high fidelity.
Because of this, the first method was found to give unreliable results during both the baseline measurement as well as during the experiments.
When transforming the images, artifacts of the method were left behind which confused the pose estimation models.
However, ViTPose still had a high recall in these cases, but SWAHR did not have any good results.
For the second method, several new styled COCO datasets were created with the newly trained style transfer models.
After training the pose estimation networks successfully, it was found that they were able to increase the performance on art collections.
To establish this, evaluation on the Human-Art dataset was very helpful, as the another method, which depended on transforming input images for evaluation, was broken.
It was found that training the models on an augmented dataset can increase the performance by at least 2\% AP.
This is in line with related works who reported a similar increase.

\section{Lessons Learned}
During the experiments, it became clear that the chosen style transfer methods didn't have the right capabilities as was first thought.
A potential culprit could be the different evaluation metrics which still do not adequately measure the similarity between images.
Evidence of this is seen in the collapse of the StarGAN network, which still gives a good performance during quantitative evaluations while qualitatively it's subpar.
While ViTPose is considered state-of-the-art, in the experiments run, it does not outperform SWAHR for any of the evaluations.
This shows again, that even though a model can be state-of-the-art in one task, this does not translate to other tasks.
During training, the pose estimation networks converged very quickly and training them for 200 epochs wasn't needed, but instead only 100 epochs would have been enough.
\section{Future Work}
To get better results, a first recommended improvement can be to use style transfer networks that have a higher fidelity.
During the discussion in section \ref{sec:improvements_discussion}, a brief look was made at several papers that evaluated the believability of different models.
They concluded that, although the older networks performed poorly, more recent models were able to perform better.
This gives hope for future development within this area.
One promising technique is stable diffusion \cite{rombach2021} which is able to synthesize high fidelity images.
It would be useful if this could be used for style transfer.
Another improvement could be the evaluation metrics for style transfer.
While there are a plethora of other metrics out there that haven't been mentioned, most of them are derivatives of the ones discussed in section \ref{sec:style_transfer_metrics}.
Their primary focus is also around generative methods and not specifically style transfer.
A specialized metric might be something worth looking at, but maybe it's enough to update the current metrics with better feature extraction methods, like a \gls{CBIR} method, which is more specialized in finding similarity.
The created datasets for style transfer can also be more refined.
Instead of having crowded images, the dataset could only focus on the human figure front and central, and crop them out as well.
While this reduces generalization, the problem is only about pose estimation.
It could even go so far that for each body part a dataset is made and style transfer is done in patches.
However, this would increase the effort that needs to be undertaken to train a network to such an extend that it might just be easier to annotate paintings for pose estimation training.
This work only focuses on realistic artworks to run pose estimation on, but this leaves out more abstract works.
Another area that deals with this is style transfer with geometric transformations.
Further research into this can extend the now limited approach.
During the experiments, the evaluations were compared with two baselines; the pre-trained model and a control model.
The control was trained the same way the styled models were and had a worse performance than the pre-trained baseline.
This means that the networks could possibly be fine-tuned to perform better.
The difference between the pre-trained and the control is as high as 6\% for SWAHR.
This is a noteworthy difference and enough to warrant further research.
All in all, there are still a lot of areas that can be improved upon.
\renewcommand\bibname{References}
\bibliography{referenties}

\end{document}